\documentclass{article}
\usepackage[spanish,activeacute,es-tabla]{babel}
\usepackage[utf8]{inputenc}
\usepackage{ifthen}
\usepackage{listings}
\usepackage{dsfont}
\usepackage{subcaption}
\usepackage{amsmath}
\usepackage[strict]{changepage}
\usepackage[top=1cm,bottom=2cm,left=1cm,right=1cm]{geometry}%
\usepackage{color}%
\newcommand{\tocarEspacios}{%
	\addtolength{\leftskip}{3em}%
	\setlength{\parindent}{0em}%
}

% Especificacion de procs

\newcommand{\In}{\textsf{in }}
\newcommand{\Out}{\textsf{out }}
\newcommand{\Inout}{\textsf{inout }}

\newcommand{\encabezadoDeProc}[4]{%
	% Ponemos la palabrita problema en tt
	%  \noindent%
	{\normalfont\bfseries\ttfamily proc}%
	% Ponemos el nombre del problema
	\ %
	{\normalfont\ttfamily #2}%
	\
	% Ponemos los parametros
	(#3)%
	\ifthenelse{\equal{#4}{}}{}{%
		% Por ultimo, va el tipo del resultado
		\ : #4}
}

\newenvironment{proc}[4][res]{%
	
	% El parametro 1 (opcional) es el nombre del resultado
	% El parametro 2 es el nombre del problema
	% El parametro 3 son los parametros
	% El parametro 4 es el tipo del resultado
	% Preambulo del ambiente problema
	% Tenemos que definir los comandos requiere, asegura, modifica y aux
	\newcommand{\requiere}[2][]{%
		{\normalfont\bfseries\ttfamily requiere}%
		\ifthenelse{\equal{##1}{}}{}{\ {\normalfont\ttfamily ##1} :}\ %
		\{\ensuremath{##2}\}%
		{\normalfont\bfseries\,\par}%
	}
	\newcommand{\asegura}[2][]{%
		{\normalfont\bfseries\ttfamily asegura}%
		\ifthenelse{\equal{##1}{}}{}{\ {\normalfont\ttfamily ##1} :}\
		\{\ensuremath{##2}\}%
		{\normalfont\bfseries\,\par}%
	}
	\renewcommand{\aux}[4]{%
		{\normalfont\bfseries\ttfamily aux\ }%
		{\normalfont\ttfamily ##1}%
		\ifthenelse{\equal{##2}{}}{}{\ (##2)}\ : ##3\, = \ensuremath{##4}%
		{\normalfont\bfseries\,;\par}%
	}
	\renewcommand{\pred}[3]{%
		{\normalfont\bfseries\ttfamily pred }%
		{\normalfont\ttfamily ##1}%
		\ifthenelse{\equal{##2}{}}{}{\ (##2) }%
		\{%
		\begin{adjustwidth}{+5em}{}
			\ensuremath{##3}
		\end{adjustwidth}
		\}%
		{\normalfont\bfseries\,\par}%
	}
	
	\newcommand{\res}{#1}
	\vspace{1ex}
	\noindent
	\encabezadoDeProc{#1}{#2}{#3}{#4}
	% Abrimos la llave
	\par%
	\tocarEspacios
}
{
	% Cerramos la llave
	\vspace{1ex}
}

\newcommand{\aux}[4]{%
	{\normalfont\bfseries\ttfamily\noindent aux\ }%
	{\normalfont\ttfamily #1}%
	\ifthenelse{\equal{#2}{}}{}{\ (#2)}\ : #3\, = \ensuremath{#4}%
	{\normalfont\bfseries\,;\par}%
}

\newcommand{\pred}[3]{%
	{\normalfont\bfseries\ttfamily\noindent pred }%
	{\normalfont\ttfamily #1}%
	\ifthenelse{\equal{#2}{}}{}{\ (#2) }%
	\{%
	\begin{adjustwidth}{+2em}{}
		\ensuremath{#3}
	\end{adjustwidth}
	\}%
	{\normalfont\bfseries\,\par}%
}

% Tipos

\newcommand{\nat}{\ensuremath{\mathds{N}}}
\newcommand{\ent}{\ensuremath{\mathds{Z}}}
\newcommand{\float}{\ensuremath{\mathds{R}}}
\newcommand{\bool}{\ensuremath{\mathsf{Bool}}}
\newcommand{\cha}{\ensuremath{\mathsf{Char}}}
\newcommand{\str}{\ensuremath{\mathsf{String}}}

% Logica

\newcommand{\True}{\ensuremath{\mathrm{true}}}
\newcommand{\False}{\ensuremath{\mathrm{false}}}
\newcommand{\Then}{\ensuremath{\rightarrow}}
\newcommand{\Iff}{\ensuremath{\leftrightarrow}}
\newcommand{\implica}{\ensuremath{\longrightarrow}}
\newcommand{\IfThenElse}[3]{\ensuremath{\mathsf{if}\ #1\ \mathsf{then}\ #2\ \mathsf{else}\ #3\ \mathsf{fi}}}
\newcommand{\yLuego}{\land _L}
\newcommand{\oLuego}{\lor _L}
\newcommand{\implicaLuego}{\implica _L}

\newcommand{\cuantificador}[5]{%
	\ensuremath{(#2 #3: #4)\ (%
		\ifthenelse{\equal{#1}{unalinea}}{
			#5
		}{
			$ % exiting math mode
			\begin{adjustwidth}{+2em}{}
				$#5$%
			\end{adjustwidth}%
			$ % entering math mode
		}
		)}
}

\newcommand{\existe}[4][]{%
	\cuantificador{#1}{\exists}{#2}{#3}{#4}
}
\newcommand{\paraTodo}[4][]{%
	\cuantificador{#1}{\forall}{#2}{#3}{#4}
}

%listas

\newcommand{\TLista}[1]{\ensuremath{seq \langle #1\rangle}}
\newcommand{\lvacia}{\ensuremath{[\ ]}}
\newcommand{\lv}{\ensuremath{[\ ]}}
\newcommand{\longitud}[1]{\ensuremath{|#1|}}
\newcommand{\cons}[1]{\ensuremath{\mathsf{addFirst}}(#1)}
\newcommand{\indice}[1]{\ensuremath{\mathsf{indice}}(#1)}
\newcommand{\conc}[1]{\ensuremath{\mathsf{concat}}(#1)}
\newcommand{\cab}[1]{\ensuremath{\mathsf{head}}(#1)}
\newcommand{\cola}[1]{\ensuremath{\mathsf{tail}}(#1)}
\newcommand{\sub}[1]{\ensuremath{\mathsf{subseq}}(#1)}
\newcommand{\en}[1]{\ensuremath{\mathsf{en}}(#1)}
\newcommand{\cuenta}[2]{\mathsf{cuenta}\ensuremath{(#1, #2)}}
\newcommand{\suma}[1]{\mathsf{suma}(#1)}
\newcommand{\twodots}{\ensuremath{\mathrm{..}}}
\newcommand{\masmas}{\ensuremath{++}}
\newcommand{\matriz}[1]{\TLista{\TLista{#1}}}
\newcommand{\seqchar}{\TLista{\cha}}

\renewcommand{\lstlistingname}{Código}
\lstset{% general command to set parameter(s)
	language=Java,
	morekeywords={endif, endwhile, skip},
	basewidth={0.47em,0.40em},
	columns=fixed, fontadjust, resetmargins, xrightmargin=5pt, xleftmargin=15pt,
	flexiblecolumns=false, tabsize=4, breaklines, breakatwhitespace=false, extendedchars=true,
	numbers=left, numberstyle=\tiny, stepnumber=1, numbersep=9pt,
	frame=l, framesep=3pt,
	captionpos=b,
}

\usepackage{titling}
\usepackage{tikz}
\usepackage{amsthm}

%Environment para tad
%///////////////////////////
\newenvironment{tad}[1]{
	\paragraph{} \vspace*{-4mm}
	\newcommand{\obs}[2]{\texttt{obs} ##1 : ##2}

	\vspace{1ex}
	\texttt{TAD} \textit{#1} $\{$
	\par
	\tocarEspacios
}
{

\hspace{2.5mm} $\}$
\vspace{2ex}
}
%///////////////////////////

\title{Soluci\'on practica 4 Algoritmos y Estructuras de Datos}
\author{Aitor}
\date{\today}

\begin{document}

\maketitle

\textbf{Ejercicio 1.}
Ya que $a$ y $b$ son reales y $i$ es entero, podemos decir que $\operatorname{def}(a) \equiv \operatorname{def}(b) \equiv \operatorname{def}(i) \equiv \text{True}$. Teniendo esto en cuenta:
\begin{itemize}
	\item [a)] $\operatorname{def}(a + b) \equiv \operatorname{def}(a) \land \operatorname{def}(1) \equiv \operatorname{def}(a) \land \text{True} \equiv \operatorname{def}(a) \equiv \text{True}$.
	\item [b)] $\operatorname{def}(a / b) \equiv \operatorname{def}(a) \land \operatorname{def}(b) \land (b \neq 0) \equiv \text{True} \land \text{True} \land (b \neq 0) \equiv (b \neq 0)$.
	\item [c)] $\operatorname{def}(\sqrt{a / b}) \equiv \operatorname{def}(a) \land \operatorname{def}(b) \land (b \neq 0) \land ((a \geq 0 \land b > 0) \lor (a \leq 0 \land b < 0)) \equiv \text{True} \land \text{True} \land (b \neq 0) \land ((a \geq 0 \land b > 0) \lor (a < 0 \land b < 0))$.
\end{itemize}

Recuerdo los axiomas vistos en clase:
\begin{enumerate}
	\item Axioma 1: $wp(x := E, Q) \equiv def(E) \yLuego Q^x_E$ es decir la wp(S,Q) donde S es equivalente a la conjuncion entre E(lo que se esta asignando) y reemplazar en Q, x por E.
	\item Axioma 2: $wp(skip, Q) \equiv Q$
	\item Axioma 3: $wp(S1 ; S2, Q) \equiv wp(S1, wp(S2, Q))$
	\item Axioma 4: Si S = if B then S1 else S2 endif entonces $wp(S,Q) \equiv def(B) \land _L \ ((B \land wp(S1,Q)) \lor (\lnot B \land wp(S2, Q)))$
\end{enumerate}

\textbf{Ejercicio 2.}\\
\begin{itemize}
	\item [a)] $wp(a: = a + 1; b:= a/2,  b \geq 0)$\\
	\textbf{Soluci\'on:}
	por axioma 3: $wp(a := a+1; b := a/2, b \geq 0) = wp(a := a+1, wp(b := a/2, b \geq 0))$\\
	por axioma 1: $wp(b := a/2, b \geq 0) \equiv \operatorname{def}(a/2) \yLuego Q^{b}_{a/2}$ (con $Q = b \geq 0$)\\
	y def(a/2) $\equiv def(a) \land def(2) \equiv \True \land \True \equiv \True$\\
	y $Q^{b}_{a/2} \equiv a/2 \geq 0 \iff a \geq 0$ luego $\operatorname{def}(a/2) \yLuego Q^{b}_{a/2} \equiv \True \land a\geq 0 \equiv a \geq 0$\\
	luego tenemos que $wp(b := a/2, b \geq 0) \equiv a \geq 0$\\
	ahora sustituyendo en la ecuacion anterior tenemos que $wp(a := a+1, wp(b := a/2, b \geq 0)) \equiv wp(a := a+1, a \geq 0)$\\
	por axioma 1: $wp(a := a+1, a \geq 0) \equiv def(a+1) \yLuego Q^{a}_{a+1} \equiv \True \yLuego a+1 \geq 0 \equiv a+1 \geq 0 \equiv a \geq -1$\\
	\textbf{Conclusi\'on} $wp(a := a+1; b := a/2, b \geq 0) \equiv a \geq -1$\\

	\item [b)]$wp(a := A[i] + 1; b := a*a, b \neq 2)$.\\
	\textbf{Soluci\'on:}
	por axioma 3: $wp(a := A[i] + 1; b := a*a, b \neq 2) \equiv wp(a := A[i] + 1, wp(b := a*a, b \neq 2))$\\
	por axioma 1: $wp(b := a*a, b \neq 2) \equiv def(a*a) \yLuego Q^{b}_{a*a}$ (con $Q = b \neq 2$)\\
	y def(a*a) $\equiv def(a) \land def(a) \equiv \True \land \True \equiv \True$\\
	y $Q^{b}_{a*a} \equiv a*a \neq 2 \equiv a \neq \sqrt{2}$ luego $def(a*a) \yLuego Q^{b}_{a*a} \equiv \True \land a \neq \sqrt{2} \equiv a \neq \sqrt{2}$\\
	luego tenemos que $wp(b := a*a, b \neq 2) \equiv a \neq \sqrt{2}$\\
	ahora sustituyendo en la ecuacion anterior tenemos que $wp(a := A[i] + 1, wp(b := a*a, b \neq 2)) \equiv wp(a := A[i] + 1, a \neq \sqrt{2})$\\
	por axioma 1: $wp(a := A[i] + 1, a \neq \sqrt{2}) \equiv
	def(A[i] + 1) \yLuego Q^{a}_{A[i] + 1} \equiv 
	def(A[i]) \yLuego Q^{a}_{A[i] + 1} \equiv \\
	(def(A) \land def(i) \yLuego 0 \leq i < |A|) \yLuego A[i] + 1 \neq \sqrt{2} 
	\equiv (0 \leq i < |A|) \yLuego A[i] \neq \sqrt{2} - 1$\\
	\textbf{Conclusi\'on} $wp(a := A[i] + 1; b := a*a, b \neq 2) \equiv (0 \leq i < |A|) \yLuego A[i] \neq \sqrt{2} - 1$\\
	
	\item [c)] $wp(a := A[i] + 1; a := b*b, a \geq 0)$.\\
	\textbf{Soluci\'on:}
	por axioma 3: $wp(a := A[i] + 1; a := b*b, a \geq 0) \equiv wp(a := A[i] + 1, wp(a := b*b, a \geq 0))$\\
	por axioma 1: $wp(a := b*b, a \geq 0) \equiv def(b*b) \yLuego Q^{a}_{b*b}$ (con $Q = a \geq 0$)\\
	y def(b*b) $\equiv def(b) \land def(b) \equiv \True \land \True \equiv \True$\\
	y $Q^{a}_{b*b} \equiv b*b \geq 0 \equiv True$ (ya que un numero multiplicado por si mismo siempre es mayor o igual a 0)\\
	luego $def(b*b) \yLuego Q^{a}_{b*b} \equiv \True \land \True \equiv \True$\\
	luego tenemos que $wp(a := b*b, a \geq 0) \equiv \True$\\
	ahora sustituyendo en la ecuacion anterior tenemos que $wp(a := A[i] + 1, wp(a := b*b, a \geq 0)) \equiv wp(a := A[i] + 1, \True)$ \textcolor{red}{???? como hago con ese true}\\

	\item [d)] $wp(a := a-b; b := a+b, a \geq 0 \land b \geq 0)$.\\
	\textbf{Soluci\'on:}
	por axioma 3: $wp(a := a-b; b := a+b, a \geq 0 \land b \geq 0) \equiv wp(a := a-b, wp(b := a+b, a \geq 0 \land b \geq 0))$\\
	por axioma 1: $wp(b := a+b, a \geq 0 \land b \geq 0) \equiv def(a+b) \yLuego Q^{b}_{a+b}$ (con $Q = a \geq 0 \land b \geq 0$) \textcolor{red}{??? como hago si tengo a y b, reemplazo  a o reemplazo  b?}\\
\end{itemize}

\textbf{Ejercicio 4.} Para los siguientes pares de programas S y postcondiciones Q
\begin{itemize}
	\item Escribir la precondici\'on m\'as d\'ebil P = wp(S, Q)
	\item Mostrar formalmente que la P elegida es correcta.
\end{itemize}
Hay que usar el axioma 4: Si S = if B then S1 else S2 endif entonces $wp(S,Q) \equiv def(B) \yLuego ((B \land wp(S1,Q)) \lor (\lnot B \land wp(S2, Q)))$

\textbf{Soluci\'on:}
\begin{enumerate}
	\item [a)] $B = a < 0$, $S1 = b := a$, $S2 = b := -a$, $Q \equiv (b = -|a|)$\\
	por axioma 4: $wp(S,Q) \equiv def(a < 0) \yLuego ((a < 0 \land wp(b := a, b = -|a|)) \lor (\lnot (a < 0) \land wp(b := -a, b = -|a|)))$\\
	$wp(b := a, b = -|a|) \equiv def(a) \yLuego Q^{b}_{a} \equiv \True \yLuego a = -|a| \equiv a = -|a|$ es decir a es negativo\\
	$wp(b := -a, b = -|a|) \equiv def(-a) \yLuego Q^{b}_{-a} \equiv \True \yLuego -a = -|a| \equiv -a = -|a| \equiv a = |a|$ es decir a es positivo\\
	luego tenemos que $wp(S,Q) \equiv def(a < 0) \yLuego ((a < 0 \land a = -|a|) \lor (\lnot (a < 0) \land a = |a|)) \equiv$\\
	$\True \yLuego ((a < 0 \land a = -|a|) \lor (\lnot (a < 0) \land a = |a|))
	\equiv ((a < 0 \land a = -|a|) \lor (\lnot (a < 0) \land a = |a|))$\\
	analicemos las dos partes de la disyuncion:
	\begin{itemize}
		\item [1)] $a < 0 \land a = -|a|$\\
		$a = -|a|$ esta diciendo que a es negativo, es decir $a<0$ pero el 0 lo cumple luego $a \leq 0$ reformulando tenemos que: $a < 0 \land a = -|a| \equiv a < 0 \land a \leq 0$ pero basta con decir $a < 0$ ya que si a = 0, a < 0 es falsa luego con chequear a < 0 estamos viendo todas las opciones
		\item [2)] $\lnot (a < 0) \land a = |a|$\\
		igual que en el item anterior $\lnot (a < 0)$ esta diciendo $a \geq 0$ y $a = |a|$ dice a es positivo o cero, luego tenemos que $ \lnot (a < 0) \land a = |a| \equiv a \geq 0 \land a \geq 0 \equiv a \geq 0$
	\end{itemize}
	entonces $((a < 0 \land a = -|a|) \lor (\lnot (a < 0) \land a = |a|)) \equiv a < 0 \lor a \geq 0$ que se cumple siempre entonces es $\equiv \True$\\
	recapitulando tenemos que: $wp(S,Q) \equiv def(a < 0) \yLuego ((a < 0 \land wp(b := a, b = -|a|)) \lor (\lnot (a < 0) \land wp(b := -a, b = -|a|))) \equiv \\
	\True \yLuego \True \equiv \True$

	\item [b)] $B = a < 0$, $S1 = b := a$, $S2 = b := -a$, $Q \equiv (b = |a|)$\\
	por axioma 4: $wp(S,Q) \equiv def(a < 0) \yLuego ((a < 0 \land wp(b := a, b = |a|)) \lor (\lnot (a < 0) \land wp(b := -a, b = |a|)))$\\
	por axioma 1: $wp(b := a, b = |a|) \equiv def(a) \yLuego Q^{b}_{a} \equiv \True \yLuego a = |a| \equiv a = |a| \equiv a \geq 0$\\
	igualmente con s2, por axioma 1: $wp(b := -a, b = |a|) \equiv def(-a) \yLuego Q^{b}_{-a} \equiv \True \yLuego -a = |a| \equiv -a = |a| \equiv a \leq 0$\\
	uniendo todo tenemos que: $wp(S,Q) \equiv def(a < 0) \yLuego ((a < 0 \land a \geq 0) \lor (\lnot (a < 0) \land a \leq 0))$\\
	y pasa lo contrario que en el item a, ahora tenemos que $(a < 0 \land a \geq 0)$ que nunca se cumple y $(\lnot (a < 0) \land a \leq 0)$ que tampoco se cumple.\\
	luego tenemos que $wp(S,Q) \equiv def(a < 0) \yLuego ((a < 0 \land a \geq 0) \lor (\lnot (a < 0) \land a \leq 0)) \equiv def(a < 0) \yLuego \False$\\
	$\equiv \True \yLuego \False \equiv \False$\\

	\item [c)] $B = i > 0$, $S1 = s[i] := 0$, $S2 = s[0] = 0$, $Q \equiv \paraTodo[unalinea]{j}{\ent}{0 \leq j < |s| \yLuego s[j] \geq 0}$
\end{enumerate}

\textbf{Ejercicio 5.}Para las siguientes especificaciones:
\begin{itemize}
	\item Poner nombre al problema que resuelven
	\item Escribir un programa S sencillo en SmallLang, sin ciclos, que lo resuelva
	\item Dar la precondici\'on m\'as d\'ebil del programa escrito con respecto a la postcondici\'on de su especificaci\'on
\end{itemize}
\textbf{Soluci\'on:}
\begin{itemize}
	\item [a])
	\begin{lstlisting}
		func agregarI-esismo(s <int>, i int, a int) int {
			a := a + s[i];
			return a
		}
	\end{lstlisting}
	\item [b])
	\begin{lstlisting}
		func positivosHastaI(s <int>, i int) bool {
			return s[i] >= 0
		}
	\end{lstlisting}
\end{itemize}

\textbf{Ejercicio 6.} Dada la poscondici\'on $Q \equiv \paraTodo[unalinea]{j}{\ent}{0 \leq j < |s| \yLuego s[j] mod 2 = 0}$ y el siguiente c\'odigo:
\begin{lstlisting}
	if (i mod 3 = 0)
		s[i] := s[i] + 6;
	else
		s[i] := i;
	endif
\end{lstlisting}
a) Demostrar que las siguientes WPs son incorrectas dando un contraejemplo:\\

\textbf{Ejercicio 7.}Dado el siguiente condicional determinar la precondici\'on m\'as d\'ebil que permite hacer valer la poscondici\'on (Q)
propuesta. Se pide:
\begin{itemize}
	\item Describir en palabras la WP esperada
	\item Derivarla formalmente a partir de los axiomas de precondici\'on m\'as d\'ebil. Para obtener el puntaje m\'aximo deber\'a simpli-
	ficarla lo m\'as posible.
\end{itemize}
a)$Q \equiv {\paraTodo[unalinea]{j}{\ent}{0 \leq j < |s| \yLuego s[j] > 0}}$
\begin{lstlisting}
	if s[i] < 0 then
		s[i] := -s[i]
	else
		s[i] := 0;
	end
\end{lstlisting}
Axioma 1: $wp(x := E, Q) \equiv def(E) \yLuego Q^x_E$ es decir la wp(S,Q) donde S es equivalente a la conjuncion entre E(lo que se esta asignando) y reemplazar en Q, x por E.\\
\textbf{Soluci\'on:}Traduciendo un poco Q dice que todos los elementos del arreglo son mayores a 0, es decir $s[j] > 0$ para todo j. Luego la wp seria:
\begin{itemize}
	\item i esta en rango asi S no se indefine
	\item el i-\'esimo valor deveria ser negativo, asi queda un numero mayor a 0 ya que si el numero es 0 o positivo, queda un 0 que no cumle Q
\end{itemize}
Ahora formalmente:\\
Hay que usar el axioma 4: Si S = if B then S1 else S2 endif entonces $wp(S,Q) \equiv def(B) \yLuego ((B \land wp(S1,Q)) \lor (\lnot B \land wp(S2, Q)))$

\begin{itemize}
	\item $B = s[i] < 0$
	\item $S1 = s[i] := -s[i]$
	\item $s2 = s[i] := 0$
	\item $Q \equiv {\paraTodo[unalinea]{j}{\ent}{0 \leq j < |s| \yLuego s[j] > 0}}$
	\item $wp(S1,Q) \equiv wp(s[i] := -s[i], Q) \equiv wp(s := setAt(s,i,-s[i]), Q)$
	\item $wp(S2,Q) \equiv wp(s[i] := 0, Q) \equiv wp(s := setAt(s,i,0), Q)$
\end{itemize}
Ahora tenemos que por axioma 4:\\
\indent$wp(S,Q) \equiv def(B) \yLuego ((B \land wp(S1,Q)) \lor (\lnot B \land wp(S2, Q)))$\\

$\equiv def(s[i] < 0) \yLuego ((s[i] < 0 \land wp(s := setAt(s,i,-s[i]), Q)) \lor (\lnot (s[i] < 0) \land wp(s := setAt(s,i,0), Q)))$\\

$\equiv def(s) \land def(i) \land 0 \leq i < |s| \yLuego\\
((s[i] < 0 \land wp(s := setAt(s,i,-s[i]), Q)) \lor (\lnot (s[i] < 0) \land wp(s := setAt(s,i,0), Q)))$\\

$\equiv_{ax1} (def(s) \land def(i)) \land 0 \leq i < |s| \yLuego \\
((s[i] < 0 \land def(setAt(s,i,-s[i])) \yLuego Q^{s}_{setAt(s,i,-s[i])}) \lor\\
(\lnot (s[i] < 0) \land def(setAt(s,i,0)) \yLuego Q^{s}_{setAt(s,i,0)}))$\\

$\equiv (True \land True) 0 \leq i < |s| \yLuego \\
((s[i] < 0 \yLuego def(s) \land def(i) \land 0 \leq i < |s| \yLuego Q^{s}_{setAt(s,i,-s[i])}) \lor\\
(\lnot (s[i] < 0) \land def(s) \land def(i) \land 0 \leq i < |s| \yLuego Q^{s}_{setAt(s,i,0)}))$\\

$\equiv 0 \leq i < |s| \yLuego \\
((s[i] < 0 \yLuego Q^{s}_{setAt(s,i,-s[i])}) \lor\\
(\lnot (s[i] < 0) \yLuego Q^{s}_{setAt(s,i,0)}))$\\

$\equiv 0 \leq i < |s| \yLuego \\
((s[i] < 0 \yLuego \paraTodo[unalinea]{j}{\ent}{0 \leq j < |s| \implicaLuego setAt(s,i,-s[i])[j] > 0}) \lor\\
(\lnot (s[i] < 0) \yLuego \paraTodo[unalinea]{j}{\ent}{0 \leq j < |s| \implicaLuego setAt(s,i,0)[j] > 0}))$\\

$\equiv 0 \leq i < |s| \yLuego \\
((s[i] < 0 \yLuego \paraTodo[unalinea]{j}{\ent}{0 \leq j < |s| \land j \neq i \implicaLuego s[j] > 0}) \land setAt(s, i, -s[i])[i] > 0 \lor\\
(\lnot (s[i] < 0) \yLuego \paraTodo[unalinea]{j}{\ent}{0 \leq j < |s| \land j \neq i \implicaLuego s[j] > 0})) \land setAt(s, i, 0)[i] > 0$\\

$\equiv 0 \leq i < |s| \yLuego \\
((s[i] < 0 \yLuego \paraTodo[unalinea]{j}{\ent}{0 \leq j < |s| \land j \neq i \implicaLuego s[j] > 0}) \land -s[i] > 0 \lor\\
(\lnot (s[i] < 0) \yLuego \paraTodo[unalinea]{j}{\ent}{0 \leq j < |s| \land j \neq i \implicaLuego s[j] > 0})) \land 0 > 0$\\

$\equiv 0 \leq i < |s| \yLuego \\
((s[i] < 0 \yLuego \paraTodo[unalinea]{j}{\ent}{0 \leq j < |s| \land j \neq i \implicaLuego s[j] > 0}) \land \textcolor{blue}{True} \lor\\
(\lnot (s[i] < 0) \yLuego \paraTodo[unalinea]{j}{\ent}{0 \leq j < |s| \land j \neq i \implicaLuego s[j] > 0})) \land \textcolor{red}{False}$\\

$\equiv 0 \leq i < |s| \yLuego \\
((s[i] < 0 \yLuego \paraTodo[unalinea]{j}{\ent}{0 \leq j < |s| \land j \neq i \implicaLuego s[j] > 0}))$\\
\qedsymbol

b) por axioma 4: $wp(S,Q) \equiv def(B) \yLuego ((B \land wp(S1,Q)) \lor (\lnot B \land wp(S2, Q)))$
\begin{itemize}
	\item $Q \equiv \existe[unalinea]{j}{\ent}{j \geq 0 \land j^2 = a}$ es decir existe un j tal que j es mayor o igual a 0 y j al cuadrado es igual a $a$
	\item $B \equiv a mod 2 = 0$ es decir a es par
	\item $S1 \equiv a := a * a$ es decir se reemplaza $a$ por $a^2$
	\item $S2 \equiv a := -|a|$ es decir si $a$ es negativo queda igual y si $a$ es positivo se reemplaza por $-a$ 
\end{itemize}

$wp(S1,Q) \equiv wp(a := a * a, Q) \equiv_{ax(1)} def(a * a) \yLuego Q^a_{a * a}\\
\equiv def(a) \yLuego \existe[unalinea]{j}{\ent}{j \geq 0 \land j^2 = a * a}\\
\equiv \True \yLuego \existe[unalinea]{j}{\ent}{j \geq 0 \land j^2 = a * a}\\
\equiv \existe[unalinea]{j}{\ent}{j \geq 0 \land j^2 = a * a}$\\
esto es True por que si $j = a$ entonces $j^2 = a^2$

$wp(S2,Q) \equiv wp(a := -|a|, Q) \equiv_{ax(1)} def(-|a|) \yLuego Q^{a}_{-|a|}\\
\equiv def(a) \yLuego \existe[unalinea]{j}{\ent}{j \geq 0 \land j^2 = -|a|}\\
\equiv \True \yLuego \existe[unalinea]{j}{\ent}{j \geq 0 \land j^2 = -|a|}$\\
esto es False por que no existe un j tal que $j^2 = -|a|$ ya que el cuadrado de un numero es positivo y -|a| es negativo\\

juntando todo tenemos que:
$wp(S,Q) \equiv def(a) \yLuego ((a mod 2 = 0 \land wp(a := a * a, Q)) \lor (\lnot (a mod 2 = 0) \land wp(a := -|a|, Q)))\\
\equiv True \yLuego ((a mod 2 = 0 \land True) \lor (\lnot (a mod 2 = 0) \land False))$ \ \ \ \ (sigue)\\
$\equiv ((a mod 2 = 0 \land True) \lor False)\\
\equiv a mod 2 = 0 \land True\\
\equiv a mod 2 = 0$\\


\end{document}