\documentclass{article}
\usepackage[spanish,activeacute,es-tabla]{babel}
\usepackage[utf8]{inputenc}
\usepackage{ifthen}
\usepackage{listings}
\usepackage{dsfont}
\usepackage{subcaption}
\usepackage{amsmath}
\usepackage[strict]{changepage}
\usepackage[top=1cm,bottom=2cm,left=1cm,right=1cm]{geometry}%
\usepackage{color}%
\newcommand{\tocarEspacios}{%
	\addtolength{\leftskip}{3em}%
	\setlength{\parindent}{0em}%
}

% Especificacion de procs

\newcommand{\In}{\textsf{in }}
\newcommand{\Out}{\textsf{out }}
\newcommand{\Inout}{\textsf{inout }}

\newcommand{\encabezadoDeProc}[4]{%
	% Ponemos la palabrita problema en tt
	%  \noindent%
	{\normalfont\bfseries\ttfamily proc}%
	% Ponemos el nombre del problema
	\ %
	{\normalfont\ttfamily #2}%
	\
	% Ponemos los parametros
	(#3)%
	\ifthenelse{\equal{#4}{}}{}{%
		% Por ultimo, va el tipo del resultado
		\ : #4}
}

\newenvironment{proc}[4][res]{%
	
	% El parametro 1 (opcional) es el nombre del resultado
	% El parametro 2 es el nombre del problema
	% El parametro 3 son los parametros
	% El parametro 4 es el tipo del resultado
	% Preambulo del ambiente problema
	% Tenemos que definir los comandos requiere, asegura, modifica y aux
	\newcommand{\requiere}[2][]{%
		{\normalfont\bfseries\ttfamily requiere}%
		\ifthenelse{\equal{##1}{}}{}{\ {\normalfont\ttfamily ##1} :}\ %
		\{\ensuremath{##2}\}%
		{\normalfont\bfseries\,\par}%
	}
	\newcommand{\asegura}[2][]{%
		{\normalfont\bfseries\ttfamily asegura}%
		\ifthenelse{\equal{##1}{}}{}{\ {\normalfont\ttfamily ##1} :}\
		\{\ensuremath{##2}\}%
		{\normalfont\bfseries\,\par}%
	}
	\renewcommand{\aux}[4]{%
		{\normalfont\bfseries\ttfamily aux\ }%
		{\normalfont\ttfamily ##1}%
		\ifthenelse{\equal{##2}{}}{}{\ (##2)}\ : ##3\, = \ensuremath{##4}%
		{\normalfont\bfseries\,;\par}%
	}
	\renewcommand{\pred}[3]{%
		{\normalfont\bfseries\ttfamily pred }%
		{\normalfont\ttfamily ##1}%
		\ifthenelse{\equal{##2}{}}{}{\ (##2) }%
		\{%
		\begin{adjustwidth}{+5em}{}
			\ensuremath{##3}
		\end{adjustwidth}
		\}%
		{\normalfont\bfseries\,\par}%
	}
	
	\newcommand{\res}{#1}
	\vspace{1ex}
	\noindent
	\encabezadoDeProc{#1}{#2}{#3}{#4}
	% Abrimos la llave
	\par%
	\tocarEspacios
}
{
	% Cerramos la llave
	\vspace{1ex}
}

\newcommand{\aux}[4]{%
	{\normalfont\bfseries\ttfamily\noindent aux\ }%
	{\normalfont\ttfamily #1}%
	\ifthenelse{\equal{#2}{}}{}{\ (#2)}\ : #3\, = \ensuremath{#4}%
	{\normalfont\bfseries\,;\par}%
}

\newcommand{\pred}[3]{%
	{\normalfont\bfseries\ttfamily\noindent pred }%
	{\normalfont\ttfamily #1}%
	\ifthenelse{\equal{#2}{}}{}{\ (#2) }%
	\{%
	\begin{adjustwidth}{+2em}{}
		\ensuremath{#3}
	\end{adjustwidth}
	\}%
	{\normalfont\bfseries\,\par}%
}

% Tipos

\newcommand{\nat}{\ensuremath{\mathds{N}}}
\newcommand{\ent}{\ensuremath{\mathds{Z}}}
\newcommand{\float}{\ensuremath{\mathds{R}}}
\newcommand{\bool}{\ensuremath{\mathsf{Bool}}}
\newcommand{\cha}{\ensuremath{\mathsf{Char}}}
\newcommand{\str}{\ensuremath{\mathsf{String}}}

% Logica

\newcommand{\True}{\ensuremath{\mathrm{true}}}
\newcommand{\False}{\ensuremath{\mathrm{false}}}
\newcommand{\Then}{\ensuremath{\rightarrow}}
\newcommand{\Iff}{\ensuremath{\leftrightarrow}}
\newcommand{\implica}{\ensuremath{\longrightarrow}}
\newcommand{\IfThenElse}[3]{\ensuremath{\mathsf{if}\ #1\ \mathsf{then}\ #2\ \mathsf{else}\ #3\ \mathsf{fi}}}
\newcommand{\yLuego}{\land _L}
\newcommand{\oLuego}{\lor _L}
\newcommand{\implicaLuego}{\implica _L}

\newcommand{\cuantificador}[5]{%
	\ensuremath{(#2 #3: #4)\ (%
		\ifthenelse{\equal{#1}{unalinea}}{
			#5
		}{
			$ % exiting math mode
			\begin{adjustwidth}{+2em}{}
				$#5$%
			\end{adjustwidth}%
			$ % entering math mode
		}
		)}
}

\newcommand{\existe}[4][]{%
	\cuantificador{#1}{\exists}{#2}{#3}{#4}
}
\newcommand{\paraTodo}[4][]{%
	\cuantificador{#1}{\forall}{#2}{#3}{#4}
}

%listas

\newcommand{\TLista}[1]{\ensuremath{seq \langle #1\rangle}}
\newcommand{\lvacia}{\ensuremath{[\ ]}}
\newcommand{\lv}{\ensuremath{[\ ]}}
\newcommand{\longitud}[1]{\ensuremath{|#1|}}
\newcommand{\cons}[1]{\ensuremath{\mathsf{addFirst}}(#1)}
\newcommand{\indice}[1]{\ensuremath{\mathsf{indice}}(#1)}
\newcommand{\conc}[1]{\ensuremath{\mathsf{concat}}(#1)}
\newcommand{\cab}[1]{\ensuremath{\mathsf{head}}(#1)}
\newcommand{\cola}[1]{\ensuremath{\mathsf{tail}}(#1)}
\newcommand{\sub}[1]{\ensuremath{\mathsf{subseq}}(#1)}
\newcommand{\en}[1]{\ensuremath{\mathsf{en}}(#1)}
\newcommand{\cuenta}[2]{\mathsf{cuenta}\ensuremath{(#1, #2)}}
\newcommand{\suma}[1]{\mathsf{suma}(#1)}
\newcommand{\twodots}{\ensuremath{\mathrm{..}}}
\newcommand{\masmas}{\ensuremath{++}}
\newcommand{\matriz}[1]{\TLista{\TLista{#1}}}
\newcommand{\seqchar}{\TLista{\cha}}

\renewcommand{\lstlistingname}{Código}
\lstset{% general command to set parameter(s)
	language=Java,
	morekeywords={endif, endwhile, skip},
	basewidth={0.47em,0.40em},
	columns=fixed, fontadjust, resetmargins, xrightmargin=5pt, xleftmargin=15pt,
	flexiblecolumns=false, tabsize=4, breaklines, breakatwhitespace=false, extendedchars=true,
	numbers=left, numberstyle=\tiny, stepnumber=1, numbersep=9pt,
	frame=l, framesep=3pt,
	captionpos=b,
}

\usepackage{titling}
\usepackage{tikz}
\usepackage{amsthm}
\usepackage{amssymb}
\usepackage{amsmath}

%Environment para tad
%///////////////////////////
\newenvironment{tad}[1]{
	\paragraph{} \vspace*{-4mm}
	\newcommand{\obs}[2]{\texttt{obs} ##1 : ##2}

	\vspace{1ex}
	\texttt{TAD} \textit{#1} $\{$
	\par
	\tocarEspacios
}
{

\hspace{2.5mm} $\}$
\vspace{2ex}
}
%///////////////////////////

\title{Soluci\'on practica 4 parte 2 Algoritmos y Estructuras de Datos}
\author{Aitor}
\date{\today}

\begin{document}

\maketitle

\newpage
\textbf{Ejercicio 1.}
\begin{itemize}
    \item [a)] Precondici\'on: $P_c \equiv \{res = 0 \land i = 0\}$
    Postcondici\'on: $Q_c \equiv \{res = \sum_{j=0}^{|s|-1} s[j] \land i = |s|\}$
    \item [b)] Falla porque al terminarse de ejecutar el ciclo, $i = |s|$ y si en el invariante se sustituye $0 \leq i \leq |s|$ por $0 \leq i < |s|$ el invariante no se cumple en la postcondicion y luego no es un invariante valido. Osea falla el punto $I \land \lnot B \implies Q_c$, ya que tendriamos $0 \leq i < |s|$ (por el invariante) y $i = |s|$ (por B) lo que nunca se puede cumplir.
    \item [c)] Falla por que al final voy a estar accediendo a un elemento que no esta.
    \item [d)] Falla $\{I \land B\}S\{I\}$ por que al terminar una iteracion res va a ser $res + s[i+1]$ envez de $res + s[i]$. 
    
    \item [e)] Demostraci\'on de correcci\'on parcial:\\
    $P_C \equiv \{res = 0 \land i = 0\}$\\
    $Q_c \equiv \{res = \sum_{j=0}^{|s|-1} s[j] \land i = |s|\}$\\
    $B \equiv \{i < |s|\}$\\
    $I \equiv \{0 \leq i \leq |s| \land res = \sum_{j=0}^{i-1} s[j]\}$

    \begin{enumerate}
        \item $P_c \implies I$\\ 
        $\iff res = 0 \land i = 0 \implies 0 \leq i \leq |s| \land res = \sum_{j=0}^{i-1} s[j]$
        $i = 0 \implies 0 \leq 0 \leq |s| \checkmark$

        \item $\{I \land B\}S\{I\}$ \\
        Como es un ciclo, para ver que la tripla de Hoare es valida uso: $P \implies wp(S, Q)$ \\
        $\{I \land B\} \implies wp(S, I)$ 
        \begin{itemize}
            \item $wp(S, I) \equiv wp(res := res + s[i]; i:= i + 1, 0 \leq i \leq |s| \land res = \sum_{j=0}^{i-1} s[j])$ \\
            $\equiv_{ax3} wp(res := res + s[i], wp(i:= i + 1, 0 \leq i \leq |s| \land res = \sum_{j=0}^{i-1} s[j]))$\\
            $\equiv_{ax1} wp(res := res + s[i], def(i + 1) \yLuego 0 \leq i + 1 \leq |s| \land res = \sum_{j=0}^{i}s[j])$\\
            $\equiv wp(res := res + s[i], 0 \leq i + 1 \leq |s| \land res = \sum_{j=0}^{i}s[j])$\\
            $\equiv_{ax1} def(res + s[i]) \yLuego 0 \leq i + 1 \leq |s| \land res + s[i] = \sum_{j=0}^{i}s[j]$\\
            $\equiv wp(s[i]) \yLuego -1 \leq i \leq |s| \land res = \sum_{j=0}^{i-1} s[j] + s[i]$\\
            $\equiv 0 \leq i < |s| \yLuego -1 \leq i \leq |s| \land res = \sum_{j=0}^{i} s[j]$\\
            $\equiv 0 \leq i < |s| \yLuego res = \sum_{j=0}^{i} s[j]$
            \item $\{I \land B\} \equiv 0 \leq i \leq |s| \land res = \sum_{j=0}^{i-1} s[j] \land i < |s|$\\
            $\equiv 0 \leq i < |s| \land res = \sum_{j=0}^{i-1} s[j]$
        \end{itemize}
        $\{I \land B\} \implies wp(S, I)$? sii \checkmark
        \item $(I \land \lnot B) \implies Q_c$
        \begin{itemize}
            \item $\{I \land \lnot B\} \equiv 0 \leq i \leq |s| \land res = \sum_{j=0}^{i-1} s[j] \land i \geq |s|$\\
            $\equiv |s| \leq i \leq |s| \land res = \sum_{j=0}^{i}s[j]$\\
            $\equiv i = |s| \land res = \sum_{j=0}^{i-1} s[j]$\\
            $\equiv Q_c$ \checkmark
        \end{itemize}    
    \end{enumerate}
    Queda probado que el ciclo es parcialmente correcto.
    
    \item [f)] Una posible funcion variante es $F_v = |s| - i$, para probar que el ciclo termina debo probar que:
    \begin{enumerate}
        \item $\{I \land B \land v_0 = fv\} S \{fv < v_0\}$
        \item $I \land fv \leq 0 \implies \lnot B$
    \end{enumerate}
\end{itemize}

\newpage
\textbf{Ejercicio 2.} 
\begin{itemize}
    \item [a)] Precondici\'on: $P_c \equiv \{n \geq 0 \land i = 0 \land res = 0\}$\\
    Postcondici\'on: $Q_c \equiv \{res = \sum_{j=0}^{n-1} (IfThenElse(j\ mod\ 2 = 0,\ j,\ 0))\}$
    \item [b)] Primero interpretemos que hace el ciclo, que es bacicamente el nombre, suma todos los numeros pares desde 0 hasta n. Esto lo hace empezando i en 0 e incrementando en 2 y sumando i todas las veces antes de incrementar.
    Uso:
    \begin{itemize}
        \item $P_c \equiv \{n \geq 0 \land i = 0 \land res = 0\}$
        \item $Q_c \equiv \{res = \sum_{j=0}^{n-1} (IfThenElse(j\ mod\ 2 = 0,\ j,\ 0))\}$
        \item $B \equiv \{i < n\}$
        \item $I \equiv \{0 \leq i \leq n + 1 \land i mod 2 = 0 \land res = \sum_{j=0}^{i-1} (IfThenElse(j\ mod\ 2 = 0,\ j,\ 0))\}$
    \end{itemize}
    Empecemos la demostraci\'on de la correcci\'on parcial:
    \begin{enumerate}
        \item $P_c \implies I$\\
        Si partimos de que $n \geq 0 \land i = 0 \land res = 0$ luego reemplazando en el invariante:\\
        $0 \leq 0 \leq n + 1 \land 0 mod 2 = 0 \land 0 = \sum_{j=0}^{0-1} (IfThenElse(j\ mod\ 2 = 0,\ j,\ 0))$\\
        $\equiv 0 \leq 0 \leq n + 1 \land 0 = 0 \land 0 = 0 \equiv \True$ \checkmark

        \item $\{I \land B\}S\{I\}$\\
        Como es un ciclo, para ver que la tripla de Hoare es valida uso: $P \implies wp(S, Q)$ 
    
        \begin{itemize}
            \item $wp(S, I)$\\
            $\equiv wp(res := res + i; i:= i + 2, 0 \leq i \leq n + 1 \land i mod 2 = 0 \land res = \displaystyle\displaystyle\sum_{j=0}^{i-1} (IfThenElse(j\ mod\ 2 = 0,\ j,\ 0)))$\\
            $\equiv_{ax3} wp(res := res + i, wp(i:= i + 2, 0 \leq i \leq n + 1 \land i mod 2 = 0 \land res = \displaystyle\sum_{j=0}^{i-1} (IfThenElse(j\ mod\ 2 = 0,\ j,\ 0))))$
            \begin{itemize}
                \item $wp(i:= i + 2, 0 \leq i \leq n + 1 \land i mod 2 = 0 \land res = \displaystyle\sum_{j=0}^{i-1} (IfThenElse(j\ mod\ 2 = 0,\ j,\ 0)))$\\
                $\equiv_{ax1} def(i+2) \yLuego 0 \leq i + 2 \leq n + 1 \land i + 2\ mod\ 2 = 0 \land res = \displaystyle\displaystyle\sum_{j =0}^{i + 1}(IfThenElse(j\ mod\ 2 = 0,\ j,\ 0))$\\
                $\equiv 0 \leq i + 2 \leq n + 1 \land i\ mod\ 2 = 0 \land res = \displaystyle\sum_{j =0}^{i + 1}(IfThenElse(j\ mod\ 2 = 0,\ j,\ 0))$
            \end{itemize}
            $\equiv_{ax1} def(res + i) \yLuego 0 \leq i + 2 \leq n + 1 \land i\ mod\ 2 = 0 \land res + i = \displaystyle\sum_{j =0}^{i + 1}(IfThenElse(j\ mod\ 2 = 0,\ j,\ 0))$\\
            $\equiv 0 \leq i + 2 \leq n + 1 \land i\ mod\ 2 = 0 \land res + i = \displaystyle\sum_{j =0}^{i + 1}(IfThenElse(j\ mod\ 2 = 0,\ j,\ 0))$\\
            $\equiv_{ax1} 0 \leq i + 2 \leq n + 1 \land i\ mod\ 2 = 0 \land res + i = \displaystyle\sum_{j =0}^{i + 1}(IfThenElse(j\ mod\ 2 = 0,\ j,\ 0))$\\
            $\equiv 0 \leq i + 2 \leq n + 1 \land i\ mod\ 2 = 0 \land res = \displaystyle\sum_{j =0}^{i + 1}(IfThenElse(j\ mod\ 2 = 0,\ j,\ 0)) - i$\\
            $\equiv_{ax1} 0 \leq i + 2 \leq n + 1 \land i\ mod\ 2 = 0 \land res = \displaystyle\sum_{j =0}^{i - 1}(IfThenElse(j\ mod\ 2 = 0,\ j,\ 0))$\\
            Ac\'a lo que hago es sacar al i de la suma (pq lo tenia que restar) entonces para hacer eso hago que la sumatoria vaya hasta el anterior a i (i - 1) y no tengo que agregar el i + 1 que saque porque como i es par, i + 1 es impar entonces la sumatoria no lo tenia en cuenta pro el if.
            \item $\{I \land B\} \equiv 0 \leq i \leq n + 1 \land i mod 2 = 0 \land res = \sum_{j=0}^{i-1} (IfThenElse(j\ mod\ 2 = 0,\ j,\ 0)) \land i < n$\\
            $\equiv 0 \leq i < n \land i mod 2 = 0 \land res = \sum_{j=0}^{i-1} (IfThenElse(j\ mod\ 2 = 0,\ j,\ 0))$ \\
            Para dejarlo igual que la wp tenqgo que ver que si $0 \leq i$ entonces $0 \leq i + 2$ y si $i < n$ entonces $i + 2 < n + 2$ y como $n + 2 \geq n + 1$ entonces $i + 2 < n + 1$ y por lo tanto $0 \leq i + 2 \leq n + 1$ ahora si lo tengo igual que la wp entonces vale. $\{I \land B\} \implies wp(S, I)$? sii \checkmark
        \end{itemize}
        \item $(I \land \lnot B) \implies Q_c$
        \begin{itemize}
            \item $\{I \land \lnot B\} \equiv 0 \leq i \leq n + 1 \land i mod 2 = 0 \land res = \sum_{j=0}^{i-1} (IfThenElse(j\ mod\ 2 = 0,\ j,\ 0)) \land i \geq n$\\
            $\equiv n \leq i \leq n + 1 \land i mod 2 = 0 \land res = \sum_{j=0}^{i-1} (IfThenElse(j\ mod\ 2 = 0,\ j,\ 0))$\\
            $\equiv i = n + 1 \land i mod 2 = 0 \land res = \sum_{j=0}^{i-1} (IfThenElse(j\ mod\ 2 = 0,\ j,\ 0))$\\
            $\equiv Q_c$ \checkmark
        \end{itemize}
    \end{enumerate}

    \item [c)] La funcion variante tiene que decrecer y ser 0 cuadno se deje de ejecutar el ciclo (se cumple $\lnot B$). La variable que cambia de valor es i (que aumenta de 2 en 2) y la variable que no cambia de valor es n. Luego podemos usar $f_v = n - i$ peeero, cuando se deje de cumplir B, $f_v$ no va a ser 0 para que esto pase hacemos $f_v = n + 2 - i$ y ahora si ya la tenemos.\\
    Para mostrar que el ciclo termina tengo que probar que:
    \begin{enumerate}
        \item $\{I \land B \land v_0 = fv\} S \{fv < v_0\}$
        Tengo que probar que $I \land B \land v_0 = f \implies wp(S, f_v < v_0)$
        \begin{itemize}
            \item $wp(S, f_v < v_0)$\\
            $\equiv wp(res := res + i; i:= i + 2, n + 2 - i < v_0)$\\
            $\equiv_{ax3} wp(res := res + i, wp(i:= i + 2, n + 2 - i < v_0))$\\
            $\equiv_{ax1} wp(res := res + i, n + 2 - (i + 2) < v_0)$\\
            $\equiv wp(res := res + i, n - i < v_0)$\\
            $\equiv_{ax1} def(res + i) \yLuego n - i < v_0$\\
            $\equiv n - i < v_0$

            \item $I \land B \land v_0 = fv$\\
            $\equiv \{0 \leq i \leq n + 1 \land i mod 2 = 0 \land res = \sum_{j=0}^{i-1} (IfThenElse(j\ mod\ 2 = 0,\ j,\ 0))\} \land i < n \land n + 2 - i = v_0$\\
            Quiero probar la implicacion entonces res no me importa:\\
            $0 \leq i \leq n + 1 \land i mod 2 = 0 \land i < n \land n + 2 - i = v_0$\\
            $\equiv 0 \leq i < n \land i mod 2 = 0 \land n + 2 - i = v_0$
        \end{itemize}
        Ahora quiero ver que:
        $ 0 \leq i < n \land i mod 2 = 0 \land n + 2 - i = v_0 \implies n - i < v_0$\\
        Que en realidad lo que quiero ver es:
        $n + 2 - i = v_0 \implies n - i < n + 2 - i$\\
        $\equiv n - i < n + 2 - i$ Que es trivialmente cierto \checkmark

        \item $I \land fv \leq 0 \implies \lnot B$
        \begin{itemize}
            \item $I \land fv \leq 0$\\ 
            $\equiv 0 \leq i \leq n + 1 \land i mod 2 = 0 \land res = \sum_{j=0}^{i-1} (IfThenElse(j\ mod\ 2 = 0,\ j,\ 0)) \land n + 1 - i \leq 0$\\
            Queiro ver que esto implica $\lnot B \equiv i \geq n$\\
            Se que $0 \leq i \leq n + 1 \land n + 1 - i \leq 0$\\
            $\equiv n + 1 \leq i \leq n + 1 \equiv n + 1 = i$\\
            Y como $n + 1 \geq n$ Se cumple \checkmark
        \end{itemize}
    \end{enumerate}
\end{itemize}


\end{document}